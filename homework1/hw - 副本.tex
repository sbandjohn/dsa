\documentclass[UTF8]{ctexart}
\usepackage{amssymb}
\usepackage{amsmath}
\usepackage{algorithm}
\usepackage{algorithmic}
\title{数算作业}
\author{林涛\\1600012773}
\begin{document}
\maketitle
\section{} %1


\section{}%1
1)

$\because$ ${\lim_{n\to +\infty}}\frac{b^n}{a^n}={\lim_{n\to +\infty}}(\frac{b}{a})^{n}=0$\\
$\therefore$ 令c = 1, $\exists$ N, s.t. $\forall$ $n>N, b^n<ca^n$




2)

3)

4)

\section{元素右移k位的算法}%3
描述:\\
1.按k将a分成m+1段,前m段每段长度为k,最后一段长度为r = n mod k,编号为0,...,m-1,m\\
2.交换第0段与第1段的位置,保持段中顺序不变。交换新的第0段与第2段,...,交换新的第0段与第m-1段。使第0段到第m-2段整体后移k位,第m-1段移到第0段的位置。\\
3.交换第m段与a[0..r-1],使得原来第m-1段的前r个元素到位。\\
4.递归地,第0段内部右移k-r位。\\

\begin{algorithm}
\caption{rightShift($a,n,k$)}
\begin{algorithmic}
\REQUIRE $a[0..n-1], n \geq 1,0\leq k \leq n$
\ENSURE shift $a$ to $a[n-k], a[n-k+1], ..., a[0], a[1], ..., a[n-k-1]$
\IF {$k==0$} \RETURN
\ENDIF
\STATE $r=n  mod  k, m = n  div  m$
\FOR {$i = 1..m-1$}
	\FOR {$j = 0..r-1$}
		\STATE swap($a[j], a[i*k+j]$)
	\ENDFOR
\ENDFOR
\IF {$r==0$} \RETURN
\ENDIF
\FOR {$i=0..r-1$}
	\STATE swap($a[i], a[m*k+i]$)
\ENDFOR	
\STATE rightShift($a,k,k-r$)
\RETURN
\end{algorithmic}
\end{algorithm}
交换次数:$F(n) = (m-1)*k + r + F(k)$, $F(1) = O(1)$
归纳地,得:$F(n) = m*k - k +O(k) + n - m*k = O(n)$

\section{判断单向链表L是否有环}%4
描述:\\
1.令N=1\\
2.从L的头出发移动N-1步,设该点为p,若遇到尾部则返回False\\
3.从p出发,移动N步,每步都判断是否回到p,是则返回True,遇到尾部返回False\\
4.N+=1,转到1\\
时间复杂度:\\
设n为L中的点数,运算次数:$F(n) = \Sigma_{N=1}^{m} 2*N = O(n^2)$, m为表头到环的距离与环的长度的较大值。
当环的长度为$n/2$,表头距离环的距离也为n/2,则$F(n) = \Sigma_{N-1}{\frac{n}{2}} 2*N = \Omega(n^2)$
因此时间复杂度为$O(n^2)$

\section{}%5

\section{用栈S1和S2模拟一个队列}%6
思想:S1中保存队列的元素,元素在队列中的顺序是从头到尾,在S1中是从栈底到栈顶。
enqueue直接PUSH到S1的栈顶。
dequeue需将S1中的元素逆序转移到S2中,使队列头的元素处在S2的栈顶,从而POP。
队列空等价于S1空。

\begin{algorithm}
\caption{enqueue(x)}
\begin{algorithmic}
\STATE {PUSH(S1,x)}
\end{algorithmic}
\end{algorithm}

\begin{algorithm}
\caption{dequeue}
\begin{algorithmic}
\STATE{//将S1的元素转移到S2,顺序变反}
\WHILE{not Sempty(S1)}
	\STATE{POP(S1,x)}
	\STATE{PUSH(S2,x)}
\ENDWHILE
\STATE{//此时S1的栈底元素在S2的栈顶}
\STATE POP(S2,x)
\WHILE{not Sempty(S2)}
	\STATE POP(S2,x)
	\STATE PUSH(S1,x)
\ENDWHILE
\STATE{//原来的S1的栈底元素已被删除,即出队}
\end{algorithmic}
\end{algorithm}

\begin{algorithm}
\caption{queue\_empty}
\begin{algorithmic}
\RETURN Sempty(S1)
\end{algorithmic}
\end{algorithm}

\section{}%7
	\subsection{火车进站台问题}
		设F(n)为n辆火车进出站台后的顺序数。有递推式:
		\begin{gather*}
			F(0)=F(1)=1\\
			F(n)=\sum_{k=1}^{k=n}F(k-1)*F(n-k)
		\end{gather*}
		证明如下:
		考虑序列1,2,...,n。首先1入栈,考虑1第几个出栈。\\
		设1第$k(1\le k \le n)$个出栈,1出栈时需保证栈中除1外没有其它元素,所以序列2,...,n进行了k-1次入栈和k-1次出栈操作。
		可用数学归纳法证明,这些出入栈操作只涉及2,...,k这k-1个元素,不涉及k+1及以后的元素。
		2,...,k进栈、出栈的方案数为F(k-1),然后1出栈,剩下n-k个元素再进栈、出栈,方案数为F(n-k)。\\
		所以F(n) = F(0)*F(n-1) + ... + F(k-1)*F(n-k) + ... + F(n-1)*F(0)
	\subsection{关于栈的性质的证明}
		$p_1,...,p_n$是合法的出栈序列 $\iff$ 不存在$i<j<k$,s.t. $p_j<p_k<p_i$。
		首先有:$p_i<p_j$$\iff$$p_i$比$p_j$先进栈\\
		先证明充分性:要证明$p_1$,$p_2$,...,$p_n$可以依次出栈\\
		首先将比$p_1$小的元素$p_{i_1},...,p_{i_k}$入栈。然后将$p_1$入栈、出栈。得到序列$p_1$。考虑此时$p_2$可能的位置:
		\begin{enumerate}
			\item 不在栈中:继续将小于$p_2$的元素入栈,然后将$p_2$入栈、出栈,得到序列$p_1$,$p_2$。
			\item 在栈顶:出栈。得到$p_1$,$p_2$。
			\item 在栈中,不在栈顶:由入栈的顺序知,栈顶元素$p_{i_k}$满足:$p_2<p_{i_k}<p_1$,但$1<2<i_k$,矛盾。这种情况不会出现。
		\end{enumerate}
		因此,总可以得到$p_1,p_2$。
		归纳地,若已得到序列$p_1,...,p_i$,此时$p_{i+1}$还未入栈或在栈顶,若在栈中不在栈顶则出现矛盾,于是可以使$p_{i+1}$紧接着出栈。从而得到序列$p_1,...,p_i,p_{i+1}$。因此$p_1$,$p_2$,...,$p_n$可以依次出栈。\\
		再证明必要性:已知$p_1$,$p_2$,...,$p_n$是合法出栈序列。假设存在$i<j<k$,使得$p_j<p_k<p_i$。\\
		$p_i$比$p_j$和$p_k$先出栈,但后入栈。说明$p_i$出栈时$p_j$和$p_k$都在栈中。$p_j<p_k$说明$p_j$比$p_k$先入栈,从而后出栈,因此$j>k$,矛盾。\\
		因此不存在这样的i,j,k。

\section{}%8
	\subsection{把12*(8*9-10)-11转化为后缀式}
	见表(\ref{tablePost})
	\begin{table}
	\centering
	\caption{转化为后缀式}
	\label{tablePost}
	\begin{tabular}{|l|l|l|}
		\hline
		符号	&栈		&后缀式\\ \hline
		12		&		&12\\ \hline
		*		&*		&12\\ \hline
		(		&*(		&12\\ \hline
		8		&*(		&12 8\\ \hline
		*		&*(*	&12 8\\ \hline
		9		&*(*	&12 8 9\\ \hline
		-		&*(-	&12 8 9 *\\ \hline
		10		&*(-	&12 8 9 * 10\\ \hline
		)		&*		&12 8 9 * 10 -\\ \hline
		-		&-		&12 8 9 * 10 - *\\ \hline
		11		&-		&12 8 9 * 10 - * 11\\ \hline
				&		&12 8 9 * 10 - * 11 - \\ \hline
	\end{tabular}
	\end{table}
	\subsection{利用栈计算上一题中得到的后缀表达式}
	见表(\ref{tableCalc})
	\begin{table}
	\centering
	\caption{计算12 8 9 * 10 - * 11 -}
	\label{tableCalc}
	\begin{tabular}{|l|l|}
		\hline
		符号	&栈		\\ \hline
		12		&12		\\ \hline
		8		&12 8		\\ \hline
		9		&12 8 9	    \\ \hline
		*		&12 72		\\ \hline
		10		&12 72 10	\\ \hline
		-		&12 62	\\ \hline
		*		&744	\\ \hline
		11		&744 11	\\ \hline
		-		&733(答案)\\ \hline
	\end{tabular}
	\end{table}
\end{document}